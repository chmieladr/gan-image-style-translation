\documentclass{article}
\usepackage[utf8]{inputenc}
\usepackage[T1]{fontenc}
\usepackage{geometry}
\usepackage{enumitem}
\usepackage{graphicx}

\geometry{margin=0.7in}

\begin{document}

\title{\textbf{Notatki}
\\ \large{\textit{Ganics}}}
\author{\textbf{Adrian Chmiel}}
\date{9 czerwca 2024}
\maketitle

\section{Introduction}
Plik ten zawiera różne notatki, które pomogły mi lepiej zrozumieć, jak działają różne GAN-y oraz jakie funkcje pełnią poszczególne warstwy. W tym dokumencie zawarte są jedynie najważniejsze informacje, które zgromadziłem podczas wykonywania projektu w celach dydaktycznych.

\section{GAN-y}
\subsection{Zwykły GAN}
\begin{itemize}
    \item jeden generator i jeden dyskryminator
    \item trening na sparowanych danych mających na celu generowanie realistycznych ogólnych obrazów
\end{itemize}

\subsection{CycleGAN}
\begin{itemize}
    \item dwa generatory i dwa dyskryminatory
    \item trening na niesparowanych danych mających na celu konwersję stylu obrazów
    \item \textbf{cykl konsystencji straty} - obraz przetłumaczony z A na B, a następnie z powrotem z B na A powinien być podobny do oryginalnego obrazu
\end{itemize}

\subsection{SRGAN}
\begin{itemize}
    \item jeden generator i jeden dyskryminator
    \item trening mający na celu generowanie wysokiej jakości obrazów
    \item \textbf{połączenia resztkowe} - przekazywanie informacji z poprzednich warstw do kolejnych z pominięciem warstw pośrednich
\end{itemize}

\subsection{Dyskryminator PatchGAN}
\begin{itemize}
    \item dyskryminator oceniający obrazy na poziomie patchy
    \item zamiast oceny całego obrazu, ocenia on poszczególne fragmenty obrazu
    \item pozwala na bardziej szczegółową ocenę obrazów
    \item często stosowany jako część \textit{SRGAN}-ów
\end{itemize}
\newpage

\section{Warstwy \textit{tf.keras}}
\subsection{Sequential}
\begin{itemize}
    \item umożliwia tworzenie modeli warstw w kolejności, w której są one dodawane
    \item prosty sposób na budowanie modeli w \textit{tf.keras}
    \item do budowania modeli, gdzie warstwy są stosowane jedna po drugiej w sposób liniowy
\end{itemize}

\subsection{Concatenate}
\begin{itemize}
    \item łączy listę tensorów wzdłuż określonego wymiaru
    \item do łączenia wyjść z różnych warstw, cechy z wcześniejszych warstw są łączone z cechami z późniejszych warstw
\end{itemize}

\subsection{Conv2D}
\begin{itemize}
    \item warstwa konwolucyjna 2D, stosowana do przetwarzania danych obrazowych
    \item stosuje filtry konwolucyjne, które przesuwają się po danych wejściowych, aby wyodrębnić cechy
    \item do ekstrakcji cech z obrazów, takich jak krawędzie, tekstury, itp.
\end{itemize}

\subsection{Conv2DTranspose}
\begin{itemize}
    \item odwrotna warstwa konwolucyjna 2D, stosowana do generowania wyższej rozdzielczości obrazu z niższej rozdzielczości
    \item do zwiększania rozdzielczości obrazów
    \item do odwracania operacji konwolucji
\end{itemize}

\subsection{ZeroPadding2D}
\begin{itemize}
    \item dodaje wypełnienie zerami wokół krawędzi danych wejściowych
    \item pozwala na kontrolowanie rozmiaru wyjściowego obrazu po zastosowaniu warstwy konwolucyjnej
    \item do utrzymania rozmiaru przestrzennego danych wejściowych po konwolucji
\end{itemize}

\subsection{LeakyReLU}
\begin{itemize}
    \item wariant funkcji aktywacji ReLU, który pozwala na niewielki gradient, gdy jednostka nie jest aktywowana (wartość ujemna)
    \item do zapobiegania problemowi "zanikających gradientów"
\end{itemize}

\subsection{InstanceNormalization / BatchNormalization}
\begin{itemize}
    \item normalizuje dane wejściowe dla każdej próbki niezależnie
    \item pozwala na stabilizację procesu trenowania i szybkie zbieżności
    \item do normalizacji cech wejściowych w sieciach neuronowych, szczególnie w modelach generatywnych i stylizacji obrazów
\end{itemize}

\subsection{ResizeLayer}
\begin{itemize}
    \item zmienia rozmiar danych wejściowych do określonego wymiaru
    \item do przeskalowywania obrazów w sieciach neuronowych (w celu standaryzacji rozmiarów obrazów wejściowych lub wyjściowych)
\end{itemize}

\end{document}